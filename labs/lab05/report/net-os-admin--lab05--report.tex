% Options for packages loaded elsewhere
% Options for packages loaded elsewhere
\PassOptionsToPackage{unicode}{hyperref}
\PassOptionsToPackage{hyphens}{url}
%
\documentclass[
  english,
  russian,
  12pt,
  a4paper,
  DIV=11,
  numbers=noendperiod]{scrreprt}
\usepackage{xcolor}
\usepackage{amsmath,amssymb}
\setcounter{secnumdepth}{5}
\usepackage{iftex}
\ifPDFTeX
  \usepackage[T1]{fontenc}
  \usepackage[utf8]{inputenc}
  \usepackage{textcomp} % provide euro and other symbols
\else % if luatex or xetex
  \usepackage{unicode-math} % this also loads fontspec
  \defaultfontfeatures{Scale=MatchLowercase}
  \defaultfontfeatures[\rmfamily]{Ligatures=TeX,Scale=1}
\fi
\usepackage{lmodern}
\ifPDFTeX\else
  % xetex/luatex font selection
\fi
% Use upquote if available, for straight quotes in verbatim environments
\IfFileExists{upquote.sty}{\usepackage{upquote}}{}
\IfFileExists{microtype.sty}{% use microtype if available
  \usepackage[]{microtype}
  \UseMicrotypeSet[protrusion]{basicmath} % disable protrusion for tt fonts
}{}
\usepackage{setspace}
% Make \paragraph and \subparagraph free-standing
\makeatletter
\ifx\paragraph\undefined\else
  \let\oldparagraph\paragraph
  \renewcommand{\paragraph}{
    \@ifstar
      \xxxParagraphStar
      \xxxParagraphNoStar
  }
  \newcommand{\xxxParagraphStar}[1]{\oldparagraph*{#1}\mbox{}}
  \newcommand{\xxxParagraphNoStar}[1]{\oldparagraph{#1}\mbox{}}
\fi
\ifx\subparagraph\undefined\else
  \let\oldsubparagraph\subparagraph
  \renewcommand{\subparagraph}{
    \@ifstar
      \xxxSubParagraphStar
      \xxxSubParagraphNoStar
  }
  \newcommand{\xxxSubParagraphStar}[1]{\oldsubparagraph*{#1}\mbox{}}
  \newcommand{\xxxSubParagraphNoStar}[1]{\oldsubparagraph{#1}\mbox{}}
\fi
\makeatother


\usepackage{longtable,booktabs,array}
\usepackage{calc} % for calculating minipage widths
% Correct order of tables after \paragraph or \subparagraph
\usepackage{etoolbox}
\makeatletter
\patchcmd\longtable{\par}{\if@noskipsec\mbox{}\fi\par}{}{}
\makeatother
% Allow footnotes in longtable head/foot
\IfFileExists{footnotehyper.sty}{\usepackage{footnotehyper}}{\usepackage{footnote}}
\makesavenoteenv{longtable}
\usepackage{graphicx}
\makeatletter
\newsavebox\pandoc@box
\newcommand*\pandocbounded[1]{% scales image to fit in text height/width
  \sbox\pandoc@box{#1}%
  \Gscale@div\@tempa{\textheight}{\dimexpr\ht\pandoc@box+\dp\pandoc@box\relax}%
  \Gscale@div\@tempb{\linewidth}{\wd\pandoc@box}%
  \ifdim\@tempb\p@<\@tempa\p@\let\@tempa\@tempb\fi% select the smaller of both
  \ifdim\@tempa\p@<\p@\scalebox{\@tempa}{\usebox\pandoc@box}%
  \else\usebox{\pandoc@box}%
  \fi%
}
% Set default figure placement to htbp
\def\fps@figure{htbp}
\makeatother



\ifLuaTeX
\usepackage[bidi=basic,provide=*]{babel}
\else
\usepackage[bidi=default,provide=*]{babel}
\fi
% get rid of language-specific shorthands (see #6817):
\let\LanguageShortHands\languageshorthands
\def\languageshorthands#1{}


\setlength{\emergencystretch}{3em} % prevent overfull lines

\providecommand{\tightlist}{%
  \setlength{\itemsep}{0pt}\setlength{\parskip}{0pt}}



 
\usepackage[backend=biber,langhook=extras,autolang=other*]{biblatex}
\addbibresource{bib/cite.bib}

\usepackage[]{csquotes}

\usepackage{indentfirst}
\usepackage{float}
\floatplacement{figure}{H}
\IfFileExists{plex-otf.sty}{
  %% Full TeXlive
  \usepackage[math,RM={Scale=0.94},SS={Scale=0.94},SScon={Scale=0.94},TT={Scale=MatchLowercase,FakeStretch=0.9},DefaultFeatures={Ligatures=Common}]{plex-otf}
}{
  %% TinyTeX
  \usepackage{libertine}
}
\KOMAoption{captions}{tableheading}
\makeatletter
\@ifpackageloaded{caption}{}{\usepackage{caption}}
\AtBeginDocument{%
\ifdefined\contentsname
  \renewcommand*\contentsname{Содержание}
\else
  \newcommand\contentsname{Содержание}
\fi
\ifdefined\listfigurename
  \renewcommand*\listfigurename{Список иллюстраций}
\else
  \newcommand\listfigurename{Список иллюстраций}
\fi
\ifdefined\listtablename
  \renewcommand*\listtablename{Список таблиц}
\else
  \newcommand\listtablename{Список таблиц}
\fi
\ifdefined\figurename
  \renewcommand*\figurename{Рисунок}
\else
  \newcommand\figurename{Рисунок}
\fi
\ifdefined\tablename
  \renewcommand*\tablename{Таблица}
\else
  \newcommand\tablename{Таблица}
\fi
}
\@ifpackageloaded{float}{}{\usepackage{float}}
\floatstyle{ruled}
\@ifundefined{c@chapter}{\newfloat{codelisting}{h}{lop}}{\newfloat{codelisting}{h}{lop}[chapter]}
\floatname{codelisting}{Список}
\newcommand*\listoflistings{\listof{codelisting}{Листинги}}
\makeatother
\makeatletter
\makeatother
\makeatletter
\@ifpackageloaded{caption}{}{\usepackage{caption}}
\@ifpackageloaded{subcaption}{}{\usepackage{subcaption}}
\makeatother
\usepackage{bookmark}
\IfFileExists{xurl.sty}{\usepackage{xurl}}{} % add URL line breaks if available
\urlstyle{same}
\hypersetup{
  pdftitle={Отчёт по лабораторной работе №6},
  pdfauthor={Кхари Жекка Кализая Арсе},
  pdflang={ru-RU},
  hidelinks,
  pdfcreator={LaTeX via pandoc}}


\title{Отчёт по лабораторной работе №6}
\usepackage{etoolbox}
\makeatletter
\providecommand{\subtitle}[1]{% add subtitle to \maketitle
  \apptocmd{\@title}{\par {\large #1 \par}}{}{}
}
\makeatother
\subtitle{Простейший вариант}
\author{Кхари Жекка Кализая Арсе}
\date{}
\begin{document}
\maketitle

\renewcommand*\contentsname{Содержание}
{
\setcounter{tocdepth}{1}
\tableofcontents
}
\listoffigures
\listoftables

\setstretch{1.5}
\chapter{Цель
работы}\label{ux446ux435ux43bux44c-ux440ux430ux431ux43eux442ux44b}

Приобретение практических навыков по установке, настройке и
администрированию СУБД MariaDB в среде виртуальной машины Rocky Linux,
включая настройку кодировки, создание базы данных, пользователей,
таблиц, резервное копирование и автоматизацию через скрипт provision.

\chapter{Задание}\label{ux437ux430ux434ux430ux43dux438ux435}

\begin{enumerate}
\def\labelenumi{\arabic{enumi}.}
\tightlist
\item
  Установить MariaDB на виртуальной машине server и выполнить базовую
  настройку безопасности.\\
\item
  Настроить кодировку символов UTF-8 для корректного хранения данных.\\
\item
  Создать базу данных \texttt{addressbook}, таблицу \texttt{city},
  добавить данные и создать пользователя с правами на работу с базой
  данных.\\
\item
  Выполнить резервное копирование базы данных и восстановление из
  резервных копий.\\
\item
  Подготовить скрипт \texttt{mysql.sh} для автоматизации установки и
  настройки MariaDB при старте ВМ.
\end{enumerate}

\chapter{Выполнение лабораторной
работы}\label{ux432ux44bux43fux43eux43bux43dux435ux43dux438ux435-ux43bux430ux431ux43eux440ux430ux442ux43eux440ux43dux43eux439-ux440ux430ux431ux43eux442ux44b}

Сначала я запустил ВМ сервер:

\begin{verbatim}
vagrant up server
\end{verbatim}

({[}рис. 01{]}).

\begin{figure}

\centering{

\includegraphics[width=0.7\linewidth,height=\textheight,keepaspectratio]{image/01.png}

}

\caption{\label{fig-01}name}

\end{figure}%

Затем я вошел под моим пользователем и открыл терминал, получив права
суперпользователя:

\begin{verbatim}
sudo -i
\end{verbatim}

({[}рис. 02{]}).

\begin{figure}

\centering{

\includegraphics[width=0.7\linewidth,height=\textheight,keepaspectratio]{image/02.png}

}

\caption{\label{fig-02}name}

\end{figure}%

Установил необходимые пакеты MariaDB:

\begin{verbatim}
dnf -y install mariadb mariadb-server
\end{verbatim}

({[}рис. 03{]}).

\begin{figure}

\centering{

\includegraphics[width=0.7\linewidth,height=\textheight,keepaspectratio]{image/03.png}

}

\caption{\label{fig-03}name}

\end{figure}%

Просмотрел конфигурационные файлы:

\begin{verbatim}
ls /etc/my.cnf.d/
cat /etc/my.cnf
\end{verbatim}

({[}рис. 04{]}).

\begin{figure}

\centering{

\includegraphics[width=0.7\linewidth,height=\textheight,keepaspectratio]{image/04.png}

}

\caption{\label{fig-04}name}

\end{figure}%

Запустил и включил MariaDB:

\begin{verbatim}
systemctl start mariadb
systemctl enable mariadb
\end{verbatim}

({[}рис. 05{]}).

\begin{figure}

\centering{

\includegraphics[width=0.7\linewidth,height=\textheight,keepaspectratio]{image/05.png}

}

\caption{\label{fig-05}name}

\end{figure}%

Проверил прослушиваемый порт:

\begin{verbatim}
ss -tulpen | grep mysql
\end{verbatim}

({[}рис. 06{]}).

\begin{figure}

\centering{

\includegraphics[width=0.7\linewidth,height=\textheight,keepaspectratio]{image/06.png}

}

\caption{\label{fig-06}name}

\end{figure}%

Запустил скрипт настройки безопасности:

\begin{verbatim}
mysql_secure_installation
\end{verbatim}

({[}рис. 07{]}).

\begin{figure}

\centering{

\includegraphics[width=0.7\linewidth,height=\textheight,keepaspectratio]{image/07.png}

}

\caption{\label{fig-07}name}

\end{figure}%

Вошел в MariaDB как root и проверил доступные базы данных:

\begin{verbatim}
mysql -u root -p
SHOW DATABASES;
\end{verbatim}

({[}рис. 08{]}).

\begin{figure}

\centering{

\includegraphics[width=0.7\linewidth,height=\textheight,keepaspectratio]{image/08.png}

}

\caption{\label{fig-08}name}

\end{figure}%

Настроил кодировку UTF-8:

\begin{verbatim}
cd /etc/my.cnf.d
touch utf8.cnf
# В utf8.cnf добавил:
# [client]
# default-character-set = utf8
# [mysqld]
# character-set-server = utf8
systemctl restart mariadb
\end{verbatim}

({[}рис. 09{]}).

\begin{figure}

\centering{

\includegraphics[width=0.7\linewidth,height=\textheight,keepaspectratio]{image/09.png}

}

\caption{\label{fig-09}name}

\end{figure}%

Создал базу данных \texttt{addressbook} и таблицу \texttt{city}:

\begin{verbatim}
mysql -u root -p
CREATE DATABASE addressbook CHARACTER SET utf8 COLLATE utf8_general_ci;
USE addressbook;
CREATE TABLE city(name VARCHAR(40), city VARCHAR(40));
\end{verbatim}

({[}рис. 10{]}).

\begin{figure}

\centering{

\includegraphics[width=0.7\linewidth,height=\textheight,keepaspectratio]{image/10.png}

}

\caption{\label{fig-10}name}

\end{figure}%

Добавил данные:

\begin{verbatim}
INSERT INTO city(name,city) VALUES ('Иванов','Москва');
INSERT INTO city(name,city) VALUES ('Петров','Сочи');
INSERT INTO city(name,city) VALUES ('Сидоров','Дубна');
\end{verbatim}

({[}рис. 11{]}).

\begin{figure}

\centering{

\includegraphics[width=0.7\linewidth,height=\textheight,keepaspectratio]{image/11.png}

}

\caption{\label{fig-11}name}

\end{figure}%

Проверил таблицу:

\begin{verbatim}
SELECT * FROM city;
\end{verbatim}

({[}рис. 12{]}).

\begin{figure}

\centering{

\includegraphics[width=0.7\linewidth,height=\textheight,keepaspectratio]{image/12.png}

}

\caption{\label{fig-12}name}

\end{figure}%

Создал пользователя и выдал права:

\begin{verbatim}
CREATE USER user@'%' IDENTIFIED BY 'password';
GRANT SELECT,INSERT,UPDATE,DELETE ON addressbook.* TO user@'%';
FLUSH PRIVILEGES;
\end{verbatim}

({[}рис. 13{]}).

\begin{figure}

\centering{

\includegraphics[width=0.7\linewidth,height=\textheight,keepaspectratio]{image/13.png}

}

\caption{\label{fig-13}name}

\end{figure}%

Вышел из MariaDB и проверил список баз и таблиц:

\begin{verbatim}
mysqlshow -u root -p
mysqlshow -u user -p addressbook
\end{verbatim}

({[}рис. 14{]}).

\begin{figure}

\centering{

\includegraphics[width=0.7\linewidth,height=\textheight,keepaspectratio]{image/14.png}

}

\caption{\label{fig-14}name}

\end{figure}%

Создал резервные копии базы данных:

\begin{verbatim}
mkdir -p /var/backup
mysqldump -u root -p addressbook > /var/backup/addressbook.sql
mysqldump -u root -p addressbook | gzip > /var/backup/addressbook.sql.gz
mysqldump -u root -p addressbook | gzip > $(date +/var/backup/addressbook.%Y%m%d.%H%M%S.sql.gz)
\end{verbatim}

({[}рис. 15{]}).

\begin{figure}

\centering{

\includegraphics[width=0.7\linewidth,height=\textheight,keepaspectratio]{image/15.png}

}

\caption{\label{fig-15}name}

\end{figure}%

Восстановил базу данных из резервной копии:

\begin{verbatim}
mysql -u root -p addressbook < /var/backup/addressbook.sql
zcat /var/backup/addressbook.sql.gz | mysql -u root -p addressbook
\end{verbatim}

({[}рис. 16{]}).

\begin{figure}

\centering{

\includegraphics[width=0.7\linewidth,height=\textheight,keepaspectratio]{image/16.png}

}

\caption{\label{fig-16}name}

\end{figure}%

Подготовил скрипт \texttt{mysql.sh} для автоматизации всех действий:

\begin{verbatim}
touch /vagrant/provision/server/mysql.sh
chmod +x /vagrant/provision/server/mysql.sh
# Внутри скрипта прописал команды установки, копирования конфигурации и резервных копий, запуск MariaDB и создание базы данных
\end{verbatim}

({[}рис. 17{]}).

\begin{figure}

\centering{

\includegraphics[width=0.7\linewidth,height=\textheight,keepaspectratio]{image/17.png}

}

\caption{\label{fig-17}name}

\end{figure}%

Добавил provision в Vagrantfile для автоматического выполнения скрипта
при старте VM.

({[}рис. 18{]}).

\begin{figure}

\centering{

\includegraphics[width=0.7\linewidth,height=\textheight,keepaspectratio]{image/18.png}

}

\caption{\label{fig-18}name}

\end{figure}%

\chapter{Выводы}\label{ux432ux44bux432ux43eux434ux44b}

В ходе лабораторной работы была успешно установлена и настроена СУБД
MariaDB на виртуальной машине Rocky Linux.\\
Была произведена настройка кодировки UTF-8, создана база данных
\texttt{addressbook} с таблицей \texttt{city}, добавлены данные, создан
пользователь с правами доступа, выполнены резервные копии и их
восстановление.\\
Также подготовлен скрипт автоматизации \texttt{mysql.sh}, обеспечивающий
повторяемость установки и настройки MariaDB при запуске ВМ.

\chapter*{Список
литературы}\label{ux441ux43fux438ux441ux43eux43a-ux43bux438ux442ux435ux440ux430ux442ux443ux440ux44b}
\addcontentsline{toc}{chapter}{Список литературы}

\printbibliography[heading=none]





\end{document}
